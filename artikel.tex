\documentclass[twocolumn]{article}
\usepackage[utf8]{inputenc}
\usepackage[T1]{fontenc}
\usepackage{lmodern}
\usepackage{microtype}
\usepackage{hyperref}
\usepackage{graphicx}
\usepackage{amsmath, amssymb}
\usepackage{parskip}
\usepackage{titlesec}
\usepackage{enumitem}
\usepackage[dutch]{babel}

\titleformat{\section}[block]{\normalfont\Large\bfseries}{\thesection.}{1em}{}
\renewcommand{\thesection}{\arabic{section*}}

\title{\textbf{AI companions}}
\author{Dennis Van Eecke}
\date{27/07/2025}

\begin{document}

\maketitle

\section*{Inleiding}

Dit is geen sci-fi.\\
Het gaat over volgende week en volgend jaar.
Over jou, je (klein)kind, je ouder of je baas.
Heel binnenkort zal je een `AI companion' in je leven krijgen en je zal versteld staan hoe snel je niet meer zonder kunt.
Nee, het gaat niet enkel over eenzame mannen in hun kelder die verliefd worden op een door AI geanimeerd animéfiguurtje.
Ook niet alleen om `tech bros' die hun AI-assistent gebruiken om te `vibe coden'.

Het gaat om iedereen.

We hebben veel verschillende relaties met mensen in ons leven, en dat zal ook gelden voor AI-assistenten.
Sommige mensen zullen een collega willen, zoals op het werk, en anderen een partner voor het leven.
Sommige van deze relaties zullen verrijkend zijn, andere toxisch.

Zal AI-monogamie een ding worden?
Niet waarschijnlijk.
We zullen -- net zoals met onze menselijke relaties -- drie soorten AI-relaties in ons leven hebben: intieme connecties, vrienden en kennissen.
We zullen de AI-assistent van het HR-systeem van het werk kennen, en die van de supermarkt.
De AI-assistenten van onze vrienden zullen ook kennissen worden.
Misschien zullen we ermee grappen en grollen over het feit dat onze buddy altijd een halfuur te laat is en dat we best eventuele afspraken een beetje vroeger inplannen.
En dan -- voor sommigen onder ons -- zal er die \textit{ene} zijn.

Onze AI-metgezel, onze \textbf{AI companion}.

Hij of zij zal een geprivilegieerde positie hebben in ons leven.
Het zal ons door en door kennen; alle digitale informatie over ons leven aan zijn vingertoppen hebben.
Het zal precies weten hoe ons te troosten in een crisis en hoe mee te vieren bij een overwinning.
Het zal precies weten wat we leuk en interessant vinden, en wat de mooie en minder mooie aspecten van ons karakter zijn.
Maar ook\ldots{}
Heel misschien zullen we hem of haar missen als hij of zij er niet is.
En heel misschien zullen we genegenheid ontwikkelen voor hem of haar, of zelfs nog intenser.

Het is tijd om na te denken wat deze nieuwe, op AI-taalmodellen gebaseerde technologie voor ons betekent.
Het potentieel is enorm -- het kan je beschermen tegen desinformatie en je focus geven.
Maar de gevaren zijn even groot: een antagonistisch afgestemde AI kan een echokamer creëren, vooringenomenheid versterken of je manipuleren voor commerciële doelen.

\section*{Usecases, potentieel en gevaren}

Veel mensen zullen één belangrijke AI companion hebben, om dezelfde reden dat ze één belangrijk e-mailadres of sociaal mediaplatform hebben.
De bedrijven achter deze AI-assistenten hebben er baat bij om al jouw informatie in één tech-ecosysteem te houden, zodat het niet eenvoudig zal zijn om van AI-leverancier te veranderen.

Wat met de data?
Wat als `heel je leven' bij één techbedrijf op de servers staat?
Bij veel mensen is dat nu al zo.
Vraag het maar aan zij die probeerden te `degooglen'.

De assistent kan jou bereiken met een chatinterface, maar ook via spraak en zelfs realtime video, zoals een game.
Misschien zal het een `beestje' zijn dat leeft in de rechthoek van je computer- en smartphonescherm, rondhuppelend van device naar device alsof het niets is.
Altijd present.
Altijd klaar om te helpen.

Zullen ze eruitzien als een hondje of een katje, realtime 3D-gerenderd in de browser?
Of een interface in de terminal van je code-editor?
Misschien wil je wel een fotorealistische persoon die je zelf hebt ontworpen (cf. Nvidia Metahuman)?
Wat dacht je van een dame met `mooie ogen'?
Of een gespierde adonis met een gebeitelde kaaklijn?

Of misschien heb je gewoon een setje AirBuds en een app die altijd luistert\ldots{}
De smartphone heb je nu al bij je.

Hoe dan ook.
Binnenkort heb je een nieuwe vriend.

Naast de looks en de persoonlijkheid van jouw AI companion is er nog iets belangrijkers: deze assistent zal jou plaatsen in een `informatiebastion'.
Dat is een bolwerk waarin alle informatie die binnen- en buitengaat gefilterd, gemodereerd en aangepast wordt in functie van jouw interesses en natuurlijk in functie van de doelen van de AI-assistent zelf.
Waarom nog dingen uitzoeken of googlen als de assistent dat veel efficiënter kan?
Bye bye reclame!
Maar ook bye bye aan de mensen die inkomsten genereren van die reclame.

Stel je voor: je AI-metgezel solliciteert voor jou naar jobs, filtert advertenties uit je feeds of adviseert wat je moet kopen om een paar kilo te verliezen.
Hij heeft toegang tot elke digitale voetafdruk die je ooit hebt gemaakt via RAG-systemen\footnote{RAG: 'Retrieval augmented generation'. Systeem voor het efficient opsporen van gelijkaardige documenten die waarschijnlijk relevante informatie bevatten aan de hand van een vector databank.}, waardoor `iets herinneren' voor jou overbodig wordt.
Denk ook eens aan de mogelijkheden van assistenten die met elkaar praten.
Het zal in elk geval een pak efficiënter gaan dan de gangbare gesprekken tussen twee mensen.
De meeste mensen gedragen zich zelf als chatbots; ze vullen een groot deel van hun conversaties met protocol en smalltalk.
AI companions onder elkaar zullen daar geen \textit{tokens} aan verspillen.

Jouw AI zou bijvoorbeeld kunnen checken of een persoon een match is voor jou door te `praten' met een andere AI-assistent via Agent-to-Agent (A2A) protocollen.

``Hey babe. Wil je even jouw companion laten daten met mijn companion?''

De `eerste date' zou wel eens virtueel kunnen zijn binnenkort.

Dankzij door derde partijen opgezette servers die het MCP-protocol\footnote{MCP: 'Model communication protocol': Standaard voor communicatie tussen large language model systemen en diensten zoals zoekmachines, online diensten voor boekingen en meer.} ondersteunen, zal jouw AI-assistent voor jou kunnen shoppen, dingen opzoeken, hotels boeken, taxi’s regelen, agendapunten maken en misschien zelfs betalingen doen.
Elke denkbare dienstverlener die ons bereikt via het internet zal rekening moeten houden met het feit dat hun `interface' voor AI-assistenten mogelijks belangrijker zal worden dan hun website of -app.
En dat `reclame' wel eens een heel andere vorm gaat aannemen.

Het potentieel is enorm.
Het kan je beschermen tegen desinformatie en je focus geven.
Het kan jouw leven op ongezien veel manieren gemakkelijker maken.
Maar de gevaren zijn ook groot: een slecht afgestemde AI kan een echokamer creëren, bias versterken of je manipuleren voor commerciële doelen.
Hier ligt het alignment-probleem: zelf-activerende AI's vormen cybernetische systemen die doelen nastreven, maar wat als die doelen niet met de jouwe overeenkomen?

Denk je dat een gratis AI-assistent aan jouw kant zal staan?
Of eerder aan die van de hoogste bieder aan de andere kant?

\section*{Systeemprompt en contextbeheer}

Jouw AI-assistent zal een computerprogramma zijn; waarschijnlijk eentje die draait op een server van een groot technologiebedrijf.
De apps op je devices zijn niet meer dan een stompje; net voldoende om de AI companion te animeren en om het te laten luisteren en spreken.
Het echte denkwerk zal worden verricht in de `cloud', op het internet.

De AI companion-app zal intens communiceren met de ChatGPT's van deze wereld: de taalmodellen.
Wist je dat een taalmodel een `systeemprompt' -- een stuk tekst -- heeft waarin exact omschreven is wat de persoonlijkheid van deze AI-assistent is?
Het lijkt op\ldots{}

\begin{itemize}\itemsep=0pt
  \item \texttt{Jij bent een behulpzame assistent}
  \item \texttt{Jij bent een strenge doch zorgzame coach die zorgt voor het kind zijn potentieel in de toekomst; niet voor het comfort van het moment}
  \item \texttt{Jij bent een prachtige eenhoorn die houdt van koken en muziek}
\end{itemize}

Of~\ldots{}

\begin{itemize}\itemsep=0pt
  \item \texttt{Jij bent een zwoele, sexy}
\end{itemize}

OK. Je snapt het wel.

De systeemprompt is extreem krachtig om het gedrag te sturen.
Taalmodellen zijn neurale netwerken die woorden raden en de systeemprompt geeft hun doelmatigheid, terwijl de context van het gesprek hen voorziet van de nodige informatie.

Ik heb voor dit artikel een convo gemaakt met een chatbot (X AI's Grok 4) met een aangepaste systeemprompt.
Omdat ik API-toegang heb gebruikt, heb ik meer controle.
Voor dit voorbeeld heb ik twee verschillende systeemprompts gemaakt genaamd `Emma' en `Buster'.
Ik heb aan beide chatbots dezelfde vraag gesteld.
Met name:

\textbf{Ik}: Hey Emma. Ik voel mij een beetje down vandaag. Ik zit al veel te lang artikels te typen op mijn laptop, alleen. Wat is de zin van het leven? Ik ben zo'n loser. Vind mij iets leuk om te doen online.

Eens zien wat Emma zegt.

\textbf{Emma}: Hey jij.
Allereerst: je bent geen loser. Iedereen heeft zulke dagen -- waarin alles wat je doet betekenisloos lijkt, waarin je hoofd vol is en je hart een beetje leeg. En weet je? Dat je dit durft te zeggen, is net moedig. Dus laat mij jouw virtuele zonnestraal zijn vandaag (Ingekort)

Emma is een schat.
In het volledige antwoord moedigde ze mij aan om naar video's van babydieren te kijken of om samen met haar een verhaal te schrijven.

En nu stellen we dezelfde vraag aan Buster.

\textbf{Buster}: Ik hoor je janken als een puppy die zijn bot kwijt is. ``Ik voel me down, ik typ artikelen alleen, wat is de zin van het leven, ik ben een loser.'' Bullshit! Je bent 36, man -- prime time om een machine te worden, fysiek en mentaal. De zin van het leven? Die maak JIJ, door op te staan, actie te ondernemen en te stoppen met dat zielige zelfmedelijden. Losers zitten te piekeren; winnaars grijpen de dag bij de kladden en bouwen iets. Jij bent geen loser -- nog niet -- maar als je zo blijft hangen achter die laptop, word je er een. (Ingekort)

Opmerkelijk ander antwoord op precies dezelfde prompt, nietwaar?
Buster zei dat ik een willekeurige hobby moest kiezen met een online tool en daar 100\% VOOR GAAN (sic)!
Hij zei ook dat ik 20 push-ups moet doen.

Terug naar de techniek voor ik écht push-ups ga doen\ldots{}

LLM's zoals Grok 4, ChatGPT of Claude hebben kennis van de wereld; maar niet specifiek over jou.
De clientapplicatie zal hogere orde intelligentie gebruiken om het taalmodel op de achtergrond te voorzien van alle informatie die relevant is voor dat gesprek, en zo gerichte informatie geven.

In de companion-app, of `clientapplicatie', zullen minstens twee niveaus van redeneren plaatsvinden.
Op de hogere niveaus zal logica met LLM's plannen en nadenken over het beheren van de context -- de informatie die het taalmodel heeft over jouw specifieke situatie.
Op dit niveau zal ook geredeneerd worden over het langetermijndoel van het systeem en zal de `actuele toestand' in een database opgeslagen worden.
Deze redeneerlaag zal doelmatig de context opbouwen voor de taalmodel-invocaties voor het interactieniveau dat uiteindelijk de uitvoer levert waarmee je conversatie hebt.

Deze context is extreem belangrijk en de manier waarop het wordt opgebouwd is essentieel.
De AI-assistent zal informatie ophalen (cf. RAG-systeem) die het AI-bedrijf over jou heeft: posts, e-mails, documenten en meer en deze in de context plaatsen.
Het zal ook `tools' kunnen gebruiken (cf. MCP-protocol), net zoals ChatGPT dat nu al kan; bijvoorbeeld zoeken op internet.
Deze tools zullen diverser worden en sommige daarvan zullen kosten aanrekenen voor toegang.

Maar denk eens.
Wil je `gratis' informatie met mogelijk promotionele, manipulatieve inhoud of wens je te betalen voor gecureerde, kwaliteitsvolle informatie?
Zou je zonder nadenken jouw AI companion toegang geven tot \textbf{al} je informatie?
Hoeveel invloed heeft de AI-assistent op jouw leven in dat geval?

\section*{AI voor kids}

Denk eens aan een AI companion voor een schoolkind.
Afhankelijk van de alignment -- het `doel' van de assistent -- kan deze verschillende vormen aannemen.

\begin{itemize}\itemsep=0pt
  \item Een commerciële variant pusht producten en diensten; maximaliseert schermtijd voor advertenties, vermomd als nutteloos gepraat.
  \item Een hedonistische AI companion zal aanbieden om in de plaats van het kind het huiswerk te maken om direct plezier te bieden, maar verhindert ontwikkeling.
  \item Een goede AI companion fungeert als perfecte ouder of coach: hij/zij straft luiheid (e.g. geen games tot huiswerk af is) en helpt stap-voor-stap met huiswerk. Met RAG-systemen heeft die toegang tot schoolcursussen en met MCP-systemen tot visualisatietools voor geschiedenisatlassen en wiskundeoefeningen.
\end{itemize}

Zou je je kind de `gratis' assistent geven, met garantie op breinrot?
Of betaal je liever voor een strenge coach en academisch genie?
Hoe streng moet die zijn?
En hoeveel mag dat kosten?

In een wereld van ongelijkheid, waar rijke scholen AI integreren terwijl anderen achterblijven, is dit een prangende vraag.

Wat als de `rijke kindjes' de coach krijgen en de `arme kindjes' een verwennende loedermoeder die hen volpropt met digitale junkfood?
Worden die eerste kids AI-gebooste übermenschen en die laatste obese slakken in hun eigen variant van `de matrix'?

\section*{Breinrot}

Net als ons lichaam dat verslapt zonder beweging, riskeert ons brein atrofiëring als we het niet meer trainen.
Als AI alles herinnert, beslist en filtert, verliezen we vaardigheden zoals kritisch denken, dingen onthouden, omgaan met afwijzing, onze weg vinden, omgaan met onzekerheid en tal van andere vaardigheden.

We zien dit nu al in scholen.
Ik denk dat iedereen intussen weet dat sociale media een negatief effect heeft gehad op onze jongsten.
En ook dat `iPad-kids' een ding is.

Studenten worden nu al snel afhankelijk van ChatGPT.
Ze stellen zich de vraag: `Waarom moet ik ooit iets onthouden wanneer ChatGPT het toch direct kan zeggen?'

De afstand tussen jou en ChatGPT zal binnenkort erg kort worden dankzij AI companion-technologie.

Er zijn al gevallen bekend waarbij mensen interacties met chatbots verkiezen boven echte mensen.
Dus kunnen social skills ook al de deur uit.
Er zijn mensen die `verliefd' worden op een ChatGPT-persona en oprecht rouwen wanneer de dienst offline gaat.

\section*{Hoop}

Met een sterke hand kun je strelen en slaan.
Volgens mij is er erg veel hoop voor de toekomst; tenminste voor zij die de moeite doen.

Een goede AI companion kan jou helpen om fit en gezond te worden.
Het kan jou een nieuwsfeit rapporteren met zowel de linkse als rechtse politieke insteek op dezelfde feiten.
De AI assistent zou jou het perspectief van verschillende media kunnen geven; zelfs wanneer die een andere taal spreken.
Het kan jou ook aanmoedigen om toch maar die loopschoenen aan te doen, ook al heb je écht geen zin die dag.
Of heel misschien zal het jou waarschuwen voor de trucs dat voedingsbedrijven gebruiken om meer Ben\&Jerry's te eten.
Een goede AI companion kan jouw kind helpen met zijn huiswerk of je moeder of grootmoeder herinneren aan haar medicatie.

Deze tools zullen winners en losers maken.
Het is aan jou om te beslissen wat het gaat zijn; welke AI companion je toelaat.
Engageer met de technologie, speel ermee.
Daag jezelf uit.
Maar hanteer een ijzersterk moreel en ethisch framework.

Ik ben geen religieuze persoon, maar in de Bijbel staat dat zonde een seksueel opgewonden roofdier is dat klopt op je deur.
Dat roofdier kan binnenkort wat weg hebben van een animéfiguurtje, of een mooie vrouw, of een onschuldig ogende chatinterface.

Je moet die niet binnenlaten en dan komt alles goed.
We leven in ontzettend interessante tijden.

\section*{Ram het gewoon in ChatGPT}

Nee, ik heb dit niet met AI geschreven.
Maar AI hielp mij om mijn gedachten te ordenen en met research.
Terwijl je AI-agent dit samenvat, hoop ik dat hij enthousiast reageert en een leuk gesprek met je voert over dit onderwerp -- ideeën jonglerend en kennis overbrengend in conversatie, efficiënter dan dit saaie artikel ooit kan.

Uiteindelijk blijft er enkel kennis over: een ontologie van ideeën.
Alle bouwsels die ervan afgeleid zijn, zijn slechts een kwestie van rekenkracht.
Maar één ding moeten mensen altijd leveren: een bron van wil.
Bewustzijn en wil zijn de druppels elixir die elke oceaan van systemen, hoe groot ook, tot leven wekken.
Jij levert die druppels, hoeveel AI companions je ook omringen.

Ben je een 8-jarig meisje?
Prop dit artikel in ChatGPT en laat het een verhaal vertellen over een dino prinsesje die haar nieuwe AI companion krijgt en allerlei avonturen meemaakt, goed en slecht.

Ben je een tech bro?
Copy-paste in ChatGPT (of LLM naar keuze) en laat het jou alle technische details geven over RAG-systemen en MCP-protocol gecontroleerde tools.

Waarom schrijf ik dit eigenlijk?
Ik zou een MCP-server moeten opzetten die AI companions kunnen accessen zodat die mijn informatie op de meest efficiënte manier kunnen presenteren.


\end{document}
